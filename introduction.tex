\addcontentsline{toc}{part}{Introduction Générale}
\markboth{\textbf{INTRODUCTION G\'EN\'ERALE}}{}
\chapter*{Introduction générale}
%\parindent=0.2cm	
Les progrès fulgurants des Technologies de l’Information et de la Communication(TIC) et les nombreuses Innovations dans les communications sans fil ainsi que le besoin de faire collaborer des objets ont conduit à un concept moderne qui est l'Internet des objets(IoT). L'avènement de l'IoT a complètement bouleversé notre quotidien. Il  nous offre une nouvelle forme d'opportunité de croissance nous permettant de délimiter la perte de temps dans la réalisation de nos tâches ainsi qu'une meilleure utilisation de nos ressources. Actuellement, il existe de nombreuses plateformes et applications pour l’IoT qui fournissent de nouveaux services en automatisant de nombreux processus notamment dans l'industrie(smart industry), la santé(smart health), le ménage, les transports(smart transport) etc.\\

Il existe plusieurs déf{\kern0pt}initions sur le concept de l’IoT, mais nous adoptons celle proposée par Weill et Souissi qui ont défini l’IoT comme « une extension de l'Internet actuel envers tout objet pouvant communiquer de manière directe ou indirecte avec des équipements électroniques eux-mêmes connectés à l'Internet. Cette nouvelle dimension de l'Internet s'accompagne de forts enjeux technologiques et économiques \cite{refdiot}. Quasiment n'importe quel appareil doté d'un bouton \og marche/arrêt \fg{} peut se connecter à l'Internet aujourd'hui, intégrant ainsi la catégorie des objets connectés \cite{ciscorefiot}.\\ 
En ce qui concerne l'IoT, les objets connectés peuvent être des objets  physiques ou virtuelles (smartphones, ordinateurs, data centers, réseaux Wi-Fi, réseaux cellulaires, puces RFID , capteurs, équipement ménager, montres, serrures, véhicules, drones, etc ) pouvant être identifiés et intégrés dans la communication des réseaux.\\
D'après la plateforme statisca \cite{refstatic} aujourd'hui le nombre d'objets connectés est estimé à 30 milliards d'objets connectés dans le monde et ce nombre atteindrait les 75 milliards  d'objets connectés en 2025.
\section*{Problématique} 
 Suite à cette popularité de l’IoT de nombreux projets et d'innovation se forment autour de cette thématique. Certains fabricants négligent l'aspect sécurité afin de limiter le temps de conception des objets connectés. ils encouragent la sortie de nouvelles solutions ou produits toujours plus rapide en considérant la sécurité comme une dernière étape avant la commercialisation d'un produit \cite{refpopIoT}.
Cependant assurer la conf{\kern0pt}identialité, la disponibilité et l'intégrité des objets connectés ainsi que les données qui y transitent sont les principales préoccupations concernant l'adoption de ce nouveau concept d'IoT. En ef{\kern0pt}fet une fois que les objets sont connectés à l'Internet, ils deviennent vulnérables à d'éventuelles at{\kern0pt}taques informat{\kern0pt}iques. l'IoT étant la prochaine génération d'Internet \cite{cisco} avec de plus en plus d'objets connectés allant de villes connectés aux bétails, la sécurité s'avère un facteur à ne pas négliger.\\

Selon 451 Research \cite{ref451rec} beaucoup d'entreprises sont toujours retissant dans l'adoption de l'IoT à cause de sa gestion de la sécurité qui est encore dans un état embryonnaire, mais 55\% des entreprises qui ont adoptées l'IoT classent la gestion de la sécurité IoT comme leur priorité absolue lors des déploiements de projets IoT au sein de leurs organisations. Les systèmes vulnérables des objets connectés  peuvent être compromis de n’importe où et utilisés pour cibler n’importe qui, raison pour laquelle la sécurité d'IoT est une préoccupation mondiale.\\

 A ce problème s’ajoute les limitations des espaces de stockages de certains objets connectés rendant la gestion de leur sécurité complexe ainsi que le problème de la disponibilité des objets connectés facilement enfreint par des attaques de types DDoS vue leur simplicité de mise en œuvre.\\
 
les problèmes de Cybersécurité en ce qui concerne les objets connectés doivent être considérés comme un challenge et un facteur primordial que doit prendre en charge les dif{\kern0pt}férents fabricants et les consommateurs (utilisateurs) avant d'adopter ce nouveau concept de réseaux d'Objets connectés.
\section*{Objectif et le Travail réalisé } 
Afin de pallier à ces problèmes, nous proposons dans ce mémoire la réalisation d'une approche résiliente pour l'ident{\kern0pt}ification et la détection des attaques DDoS dans les réseaux IoT en vue de minimiser les intrusions.\\ Notre choix du DDoS s'explique du fait qu'il utilise les objets connectés non sécurisés pour sa mise en œuvre \cite{8355541} \cite{inproceedings} et est probablement considéré comme l'une des menaces la plus courante et la plus dangereuse visant l'IoT vis à vis de l'explosion exponentielle du nombre d'objets connectés impliquant une augmentation colossale du nombre de Botnets à venir \cite{refbotnet}.\\
L'objectif essentiel de notre travail est la réalisation d'un système de détection d'intrusion. Pour se faire nous allons utiliser les techniques d’apprent{\kern0pt}issage automatique basé sur le Deep Learning ainsi que les frameworks DL4J et JavaFX pour l'implémentation. Cette approche inclue l'utilisation séquentielle de l'auto-encodeur(AE) et de réseaux de neurones profonds(DNN).
\section*{Plan du document : }
Ce document est organisé en deux parties :
\begin{description}
\item[\textbf{La première partie :}] présente l'état de l'art sur les dif{\kern0pt}férents domaines entrant en jeu dans le cadre de ce mémoire . Elle est composée de trois chapitres à savoir l'Internet des objets(l'IoT), la sécurité informat{\kern0pt}ique et l'attaque par déni de service distribué (DDoS).\\
Dans le premier chapitre nous présentons le concept de L'Internet des Objets ou nous détaillons ces caractéristiques, son architecture ainsi que ces dif{\kern0pt}férentes applications.\\
Dans le deuxième chapitre nous introduisons les concepts élémentaires de la sécurité informat{\kern0pt}ique ou nous présentons les applications de la sécurité, les dif{\kern0pt}férents types d'attaques ainsi que les dif{\kern0pt}férents services et mécanismes de sécurités.\\
Après avoir introduit l'at{\kern0pt}taque de type DDoS nous détaillons en profondeur sa composition, sa mise en œuvre ainsi que son impact sur l'IoT dans le chapitre trois 
\item[\textbf{La deuxième partie : }] présente les contributions essentielles apportées. Elle est composée d'un seul chapitre à savoir les systèmes de détections  d'intrusions basés sur le deep learning. Dans ce quatrième chapitre nous présentons les systèmes de détections d'intrusion, les dif{\kern0pt}férents aspects des systèmes de détection d'intrusion, le Deep Learning et ses dif{\kern0pt}férentes méthodes. Et par la suite nous réalisons notre approche en appliquant les modèles AE et DNN du Deep Learning.
\end{description}
Enfin, nous clôturons notre modeste travail par une conclusion générale ainsi que les dif{\kern0pt}férentes perspectives à développer dans l'avenir pour toute éventuelle amélioration de ce travail.
